\documentclass[a4paper, 12pt]{article} % 12pt, A4 page

% packages
\usepackage{amsmath} % math symbols
\usepackage{amsfonts} % 〃    〃
\usepackage{xcolor} % text coloring
\usepackage{setspace} % spacing
\usepackage{cite}
\usepackage[nottoc,numbib]{tocbibind} % somehow makes the references appear in ToC

% custom commands
\newcommand{\fatf}{\mathbf{F}} % for Vector field F
\newcommand{\definedas}{\stackrel{\Delta}{=}} % definitions
% > derivatives
\newcommand{\der}[2]{\frac{\mathrm{d} #1}{\mathrm{d} #2}} % derivative
\newcommand{\partialder}[2]{\frac{\partial #1}{\partial #2}} % partial derivative
\newcommand{\materialder}[2]{\frac{\mathrm{D} #1}{\mathrm{D} #2}} % material derivative
% > vector calculus operators
\DeclareMathOperator{\gradient}{\nabla} % gradient operator
\DeclareMathOperator{\divergence}{\nabla\cdot} % divergence operator
\DeclareMathOperator{\curl}{\nabla\times} % divergence operator

% formatting
\doublespacing % double spacing
\pagenumbering{arabic} % page numbering

\begin{document}
\begin{titlepage} % title page
	\begin{center}
		\vspace*{0.5cm}
		\Large
		\textbf{How can vector calculus be used to model and analyze incompressible fluid flow in two-dimensional spaces, and what insights can this provide about real-world fluid systems?}

		\vspace{1.5cm}
		\large
		\textbf{Mathematics AA HL}

		\vspace{9cm}\color{darkgray}
		Word Count: 86 % (maximum 4000)
	\end{center}
\end{titlepage}

\tableofcontents\newpage % table of contents

% INTRODUCTION
\section{Introduction}
% > background
\subsection{Scope}
This essay will for simplicity's sake only cover fluid flow in $\mathbb{R}^2$ spaces; an alaysis of fluid flow in $\mathbb{R}^3$ spaces would be orders of magnitude more complex.
Furthermore, only incompressible fluids (any fluid such that $\divergence\fatf=0$), such that there are no sinks nor wells, will be analyzed. Test

% > background
\subsection{Background}
Vector calculus is the mathematical study of applying multi-variable calculus to vector valued functions, often for the spaces $\mathbb{R}^2$ and $\mathbb{R}^3$.

$$f:\mathbb{R}^n\rightarrow\mathbb{R}^n\ni n>1,n\in\mathbb{Z}$$
\begin{equation}
	\left.\leadsto\nabla=
	\left[\begin{matrix}
		\partialder{}{x} \\
		\partialder{}{y} \\
		\partialder{}{z} \\
		\vdots 	         \\
	\end{matrix}\right]
	\color{gray}\right\} \color{gray}n\text{ times}\color{black}
\end{equation}
\begin{equation}
	\materialder{f}{t}\definedas\partialder{f}{t}+\vec{v}\cdot\nabla f
\end{equation}
$$\vec{v_1}\otimes\vec{v_2}$$

% > fluid dynamics
\subsection{Fluid dynamics}
An incompressible fluid is any fluid such that $\divergence\fatf=0$, which is to say that the divergence of the fluid is 0.

% > the navier-stokes equations
\subsection{The Navier-Stokes equations}
\begin{equation} % navier-stokes
	\materialder{f}{\mathbf{t}}=\iiint\limits_{V}(\materialder{\rho}{\mathbf{t}}+\rho(\nabla\cdot u))dV
\end{equation}
Lorem ipsum dolor sit amet \cite{peyret2012computational}

% references
\newpage
\bibliographystyle{apalike}
\bibliography{sources}
\end{document}