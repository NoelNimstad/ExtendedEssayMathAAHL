\section{Introduction}
Fluid dynamics is today a cornerstone to several fields of study, including ærospace engineering and meteorology.
Real world fluid behaviour is intricate and complex. Therefore, to gain insights into the governing principles of
fluid flow, simplified and idealized models are used. This essay investigates the application of vector calculus
to model and analyse steady, inviscid, and incompressible fluid flow in two-dimensional spaces around a circular
obstacle. These idealizations allow for the derivation of some of fluid dynamic's key mathematical formulæ and
provides a foundation for understanding less idealized fluids. The essay will address the question: "\researchquestion".

% Aim & scope
\subsection{Aim \& scope}
This essay will for simplicity's sake only cover fluid flow around circular obstacles in $\mathbb{R}^2$ spaces; an alaysis of fluid flow in $\mathbb{R}^3$ spaces would be much more complex.
Furthermore, only incompressible fluids sans sinks and sources ($\fatf\ni\divergence\fatf=0$), will be analyzed.

Most of the analysis will take place using Green's theorem\referto{sec:greenstheorem}.

% Background
\subsection{Background}
\subsubsection{Notation}
In this paper, the gradient, divergence and curl operators will be denoted using their explicit $\nabla$ forms as follows:
$$\begin{matrix}
	\mathrm{grad}\ \fatf&\equiv&\grad{\fatf}\\
	\mathrm{div}\ \fatf&\equiv&\divergence\fatf\\
	\mathrm{curl}\ \fatf&\equiv&\curl\fatf
\end{matrix}$$
The directional vector will also be denoted using $\nabla$ as $\nabla_{\vec{v}}\fatf$.

For the purposes of clarity, vectors in cartesian systems will be denoted $\begin{pmatrix}x\\y\end{pmatrix}$ whilst vectors in polar systems will be denoted as $\langle r,\theta\rangle$

To ensure point-uniqueness within polar systems, all polar coordinates are restricted such as $r\geq0,\,\theta\in[0,2\pi)$.