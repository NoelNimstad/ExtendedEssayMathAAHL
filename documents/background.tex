\section{Vector calculus}
\subsection{The fundamentals of vector calculus}

\begin{defn}
	\definedterm{Partial derivatives} are an extension of single-variable derivatives in which all variables save
	the one being differentiated by are treated as constants \cite{MORTIMER201389}. A formal definition of the
	partial derivative of some function $f$ with respect to a parameter $x_n$ can be expressed as:
	\begin{equation}
		\pdv{f}{x_n}=\lim_{\delta\rightarrow0}\frac{f(x_1,x_2,\cdots,x_n+\delta,\cdots)-f(x_1,x_2,\cdots,x_n,\cdots)}{\delta}
	\end{equation}
	Partial derivatives allow for the analysis of how multi-variable functions such as scalar- or vector fields change
	with respect to just one spatial dimension. For example, consider the function $f(x,y)=x^2y+\sin(x)\sin y$:
	\begin{align*}
		\pdv{f}{x}=2xy+\cos(x)\sin y&&\pdv{f}{y}=x^2+\sin(x)\cos y
	\end{align*}
	$n$-th order partial derivatives are denoted, similarly to normal calculus, as
	$$
	\pdv[n]{f}{x}=\lim_{\delta_1\rightarrow0}\lim_{\delta_2\rightarrow0}\hdots\lim_{\delta_n\rightarrow0}\frac{f(x_1,x_2,\cdots,x_n+\delta,\cdots)-f(x_1,x_2,\cdots,x_n,\cdots)}{\delta}
	$$
\end{defn}
\begin{defn}
	\definedterm{Mixed partial derivatives} are partial derivatives of a function taken with respect to multiple
	variables \cite{GARRETT2015377}. This is denoted as
	$$
	\pdv{f}{x}{y}\equiv\pdv{x}\pdv{f}{y}
	$$
	where both $\alpha$ and $\beta$ are parameters of $f$.
\end{defn}
% \begin{lemma}[Clairaut's theorem]
% 	The order of mixed partial derivatives has no effect on the outcome of the derivative.
% 	\begin{proof}
% 		Let $f:\alpha,\beta\mapsto\mathbb{R}^n,n\in\mathbb{N}$ be a function continuously differentiable over both $\alpha$ and $\beta$.
% 		\begin{align*}
% 			\pdv{f}{\alpha}&=\lim_{\delta_\alpha\rightarrow0}\frac{f(\alpha+\delta_\alpha,\beta)-f(\alpha,\beta)}{\delta_\alpha}\\
% 			\pdv{f}{\alpha}{\beta}&=\lim_{\delta_\beta\rightarrow0}\lim_{\delta_\alpha\rightarrow0}\frac{\frac{f(\alpha+\delta_\alpha,\beta+\delta_\beta)-f(\alpha,\beta+\delta_\beta)}{\delta_\alpha}-\frac{f(\alpha+\delta_\alpha,\beta)-f(\alpha,\beta)}{\delta_\alpha}}{\delta_b}\\
% 			&=\lim_{\delta_\beta\rightarrow0}\lim_{\delta_\alpha\rightarrow0}\frac{f(\alpha+\delta_\alpha,\beta+\delta_\beta)-f(\alpha,\beta+\delta_\beta)-[f(\alpha+\delta_\alpha,\beta)-f(\alpha,\beta)]}{\delta_\alpha\delta\beta}\\
% 			&=\lim_{\delta_\beta\rightarrow0}\lim_{\delta_\alpha\rightarrow0}\frac{f(\alpha+\delta_\alpha,\beta+\delta_\beta)-f(\alpha+\delta_\alpha,\beta)-f(\alpha,\beta+\delta_\beta)+f(\alpha,\beta)}{\delta_\alpha\delta_\beta}
% 		\end{align*}
% 		This process is thus clearly reversible,
% 		$$
% 		\therefore\pdv{f}{\alpha}{\beta}\equiv\pdv{f}{\beta}{\alpha}
% 		$$
% 	\end{proof}
% \end{lemma}
\begin{lemma}[Clairaut's theorem]
	The order of mixed partial derivatives has no effect on the outcome of the derivative.
	\begin{proof}
		Let a rectangular region be bound by the points $W\langle \alpha_0,\beta_0\rangle,X\langle \alpha_1,\beta_0\rangle,
		Y\langle \alpha_1,\beta_1\rangle$ and $Z\langle \alpha_0,\beta_1\rangle$. Let $Q$ represent the difference in the
		change of values at the regions of the function given at $f(\alpha,\beta_1)$ and $f(\alpha,\beta_2)$ on the
		interval $[\alpha_0,\alpha_1]$.
		$$
		Q=[f(\alpha_1,\beta_1)-f(\alpha_0,\beta_1)]-[f(\alpha_1,\beta_0)-f(\alpha_0,\beta_0)]
		$$
		The mean value theorem (MVT) states that on both regions $\vec{WX}$ and $\vec{ZY}\,\exists\,\xi_0$ and $\xi_1\ni$
		\begin{align*}
			\eval{\pdv{f}{\alpha}}_{\xi_0}&=\frac{f(\alpha_1,\beta_0)-f(\alpha_0,\beta_0)}{\lVert\vec{XY}\rVert}=\frac{f(\alpha_1,\beta_0)-f(\alpha_0,\beta_0)}{\alpha_1-\alpha_0}\\
			\eval{\pdv{f}{\alpha}}_{\xi_1}&=\frac{f(\alpha_1,\beta_1)-f(\alpha_0,\beta_1)}{\lVert\vec{ZY}\rVert}=\frac{f(\alpha_1,\beta_1)-f(\alpha_0,\beta_1)}{\alpha_1-\alpha_0}
		\end{align*}
		Thus $Q$ can be expressed as
		\begin{align*}
			Q&=\left(\eval{\pdv{f}{\alpha}}_{\xi_0}\left(\alpha_1-\alpha_0\right)\right)-\left(\eval{\pdv{f}{\alpha}}_{\xi_1}\left(\alpha_1-\alpha_0\right)\right)\\
			&=\left(\eval{\pdv{f}{\alpha}}_{\xi_0}-\eval{\pdv{f}{\alpha}}_{\xi_1}\right)\left(\alpha_1-\alpha_0\right)
		\end{align*}
		Now let $R$ be the equivalent of $Q$ in the direction of $\beta$,
		$$
			R=[f(\alpha_1,\beta_1)-f(\alpha_1,\beta_0)]-[f(\alpha_0,\beta_1)-f(\alpha_0,\beta_0)]
		$$
		The MVT asserts that on the regions $\vec{ZW}$ and $\vec{ZY}\,\exists\,\zeta_0$ and $\zeta_1\ni$
		\begin{align*}
			\eval{\pdv{f}{\beta}}_{\zeta_0}&=\frac{f(\alpha_0,\beta_1)-f(\alpha_0,\beta_0)}{\lVert\vec{ZW}\rVert}=\frac{f(\alpha_0,\beta_1)-f(\alpha_0,\beta_0)}{\beta_1-\beta_0}\\
			\eval{\pdv{f}{\beta}}_{\zeta_1}&=\frac{f(\alpha_1,\beta_1)-f(\alpha_1,\beta_0)}{\lVert\vec{ZY}\rVert}=\frac{f(\alpha_1,\beta_1)-f(\alpha_1,\beta_0)}{\beta_1-\beta_0}
		\end{align*}
		Thus $R$ can be expressed as
		\begin{align*}
			R&=\left(\eval{\pdv{f}{\beta}}_{\zeta_0}\left(\beta_1-\beta_0\right)\right)-\left(\eval{\pdv{f}{\beta}}_{\zeta_1}\left(\beta_1-\beta_0\right)\right)\\
			&=\left(\eval{\pdv{f}{\beta}}_{\zeta_0}-\eval{\pdv{f}{\beta}}_{\zeta_1}\right)\left(\beta_1-\beta_0\right)
		\end{align*}
		Rearranging $Q$ and $R$,
		\begin{align*}
			Q&=[f(\alpha_1,\beta_1)-f(\alpha_0,\beta_1)]-[f(\alpha_1,\beta_0)-f(\alpha_0,\beta_0)]\\
			&=f(\alpha_1,\beta_1)-f(\alpha_1,\beta_0)-f(\alpha_0,\beta_1)+f(\alpha_0,\beta_0)\\
			&=[f(\alpha_1,\beta_1)-f(\alpha_1,\beta_0)]-[f(\alpha_0,\beta_1)-f(\alpha_0,\beta_0)]=R\\
			\therefore Q&=R
		\end{align*}
		Thus
		\begin{align*}
			\left(\eval{\pdv{f}{\alpha}}_{\xi_0}-\eval{\pdv{f}{\alpha}}_{\xi_1}\right)\left(\alpha_1-\alpha_0\right)&=\left(\eval{\pdv{f}{\beta}}_{\zeta_0}-\eval{\pdv{f}{\beta}}_{\zeta_1}\right)\left(\beta_1-\beta_0\right)\\
			\frac{\eval{\pdv*{f}{\alpha}}_{\xi_0}-\eval{\pdv*{f}{\alpha}}_{\xi_1}}{\beta_1-\beta_0}&=\frac{\eval{\pdv*{f}{\beta}}_{\zeta_0}-\eval{\pdv*{f}{\beta}}_{\zeta_1}}{\alpha_1-\alpha_0}
		\end{align*}
	\end{proof}
\end{lemma}
\begin{defn}
	The \definedterm{nabla} operator $\nabla$ is a vector containing one partial derivative for each parameter of
	the function applied to \cite{RAPP2017137}. For some function $f:\mathbb{R}^n\rightarrow\mathbb{R}$, $\nabla f$ would be given by:
	$$
	\nabla f=\begin{bmatrix}
		\pdv*{f}{x_1}\\
		\pdv*{f}{x_2}\\
		\vdots\\
		\pdv*{f}{x_n}
	\end{bmatrix}
	$$
	The nabla of a function will be the vector which points in the direction of the greatest change for the function
	at the point evaluated at.
\end{defn}

