\section{Vector calculus}
\subsection{The fundamentals of vector calculus}

\begin{defn}
	\definedterm{Partial derivatives} are an extension of single-variable derivatives in which all variables save
	the one being differentiated by are treated as constants \cite{MORTIMER201389}. A formal definition of the
	partial derivative of some function $f$ with respect to a parameter $x_n$ can be expressed as:
	\begin{equation}
		\pdv{f}{x_n}=\lim_{\delta\rightarrow0}\frac{f(x_1,x_2,\cdots,x_n+\delta,\cdots)-f(x_1,x_2,\cdots,x_n,\cdots)}{\delta}
	\end{equation}
	Partial derivatives allow for the analysis of how multi-variable functions such as scalar- or vector fields change
	with respect to just one spatial dimension. For example, consider the function $f(x,y)=x^2y+\sin(x)\sin y$:
	\begin{align*}
		\pdv{f}{x}=2xy+\cos(x)\sin y&&\pdv{f}{y}=x^2+\sin(x)\cos y
	\end{align*}
	$n$-th order partial derivatives are denoted, similarly to normal calculus, as
	$$
	\pdv[n]{f}{x}=\lim_{\delta_1\rightarrow0}\lim_{\delta_2\rightarrow0}\hdots\lim_{\delta_n\rightarrow0}\frac{f(x_1,x_2,\cdots,x_n+\delta,\cdots)-f(x_1,x_2,\cdots,x_n,\cdots)}{\delta}
	$$
\end{defn}
\begin{defn}
	\definedterm{Mixed partial derivatives} are partial derivatives of a function taken with respect to multiple
	variables \cite{GARRETT2015377}. This is denoted as
	$$
	\pdv{f}{\alpha}{\beta}\equiv\pdv{\beta}\pdv{f}{\alpha}
	$$
	where both $\alpha$ and $\beta$ are parameters of $f$.
\end{defn}
\begin{lemma}[Clairaut's theorem]
	Let $f(\alpha,\beta)$ be a function of two parameters $\alpha$ and $\beta$. If the mixed partial derivatives $\pdv{f}{\alpha}{\beta}$
	and $\pdv{f}{\beta}{\alpha}$ exist and are continuous in the open disk $\mathbb{D}_\delta(\point{\alpha_0}{\beta_0})$ centred at
	$\point{\alpha_0}{\beta_0}$ with radius $\delta>0$, then
	$$
		\eval{\pdv{f}{\alpha}{\beta}}_{\point{\alpha_0}{\beta_0}}=\eval{\pdv{f}{\beta}{\alpha}}_{\point{\alpha_0}{\beta_0}}
	$$
	\cite{GARRETT2015377}
	\begin{proof}
		Let $\point{\alpha_0}{\beta_0}$ be a point in the domain of $f$. Consider a rectangular region bound by the
		points $W\point{\alpha_0}{\beta_0},X\point{\alpha_1}{\beta_0}, Y\point{\alpha_1}{\beta_1}$  and $Z\point{\alpha_0}{\beta_1}$. 
		$\pdv{f}{\alpha}$ and $\pdv{f}{\beta}$ exist in a neighbourhood of this rectangle, and the mixed partial derivatives 
		$\pdv{f}{\beta}{\alpha}$ and $\pdv{f}{\alpha}{\beta}$ exist and are continuous in this neighbourhood. Let $Q$ be such that
		$$
		Q=[f(\alpha_1,\beta_1)-f(\alpha_0,\beta_1)]-[f(\alpha_1,\beta_0)-f(\alpha_0,\beta_0)]
		$$
		According to the mean value theorem (MVT) $\exists\,\xi_0,\xi_1\in[\alpha_0,\alpha_1]\ni$
		\begin{align*}
			\eval{\pdv{f}{\alpha}}_{\point{\xi_0}{\beta_0}}&=\frac{f(\alpha_1,\beta_0)-f(\alpha_0,\beta_0)}{\alpha_1-\alpha_0}\\
			\eval{\pdv{f}{\alpha}}_{\point{\xi_1}{\beta_1}}&=\frac{f(\alpha_1,\beta_1)-f(\alpha_0,\beta_1)}{\alpha_1-\alpha_0}
		\end{align*}
		Thus $Q$ can be expressed as
		\begin{align*}
			Q&=\left(\eval{\pdv{f}{\alpha}}_{\point{\xi_0}{\beta_0}}\left(\alpha_1-\alpha_0\right)\right)-\left(\eval{\pdv{f}{\alpha}}_{\point{\xi_1}{\beta_1}}\left(\alpha_1-\alpha_0\right)\right)\\
			&=\left(\eval{\pdv{f}{\alpha}}_{\point{\xi_0}{\beta_0}}-\eval{\pdv{f}{\alpha}}_{\point{\xi_1}{\beta_1}}\right)\left(\alpha_1-\alpha_0\right)
		\end{align*}
		Now let $R$ be the equivalent of $Q$ in the direction of $\beta$,
		$$
			R=[f(\alpha_1,\beta_1)-f(\alpha_1,\beta_0)]-[f(\alpha_0,\beta_1)-f(\alpha_0,\beta_0)]
		$$
		By the MVT $\exists\,\zeta_0,\zeta_1\in[\beta_0,\beta_1]\ni$
		\begin{align*}
			\eval{\pdv{f}{\beta}}_{\point{\alpha_0}{\zeta_0}}&=\frac{f(\alpha_0,\beta_1)-f(\alpha_0,\beta_0)}{\beta_1-\beta_0}\\
			\eval{\pdv{f}{\beta}}_{\point{\alpha_1}{\zeta_1}}&=\frac{f(\alpha_1,\beta_1)-f(\alpha_1,\beta_0)}{\beta_1-\beta_0}
		\end{align*}
		Thus $R$ can be expressed as
		\begin{align*}
			R&=\left(\eval{\pdv{f}{\beta}}_{\point{\alpha_0}{\zeta_0}}\left(\beta_1-\beta_0\right)\right)-\left(\eval{\pdv{f}{\beta}}_{\point{\alpha_1}{\zeta_1}}\left(\beta_1-\beta_0\right)\right)\\
			&=\left(\eval{\pdv{f}{\beta}}_{\point{\alpha_0}{\zeta_0}}-\eval{\pdv{f}{\beta}}_{\point{\alpha_1}{\zeta_1}}\right)\left(\beta_1-\beta_0\right)
		\end{align*}
		Rearranging $Q$ and $R$,
		\begin{align*}
			Q&=[f(\alpha_1,\beta_1)-f(\alpha_0,\beta_1)]-[f(\alpha_1,\beta_0)-f(\alpha_0,\beta_0)]\\
			&=f(\alpha_1,\beta_1)-f(\alpha_1,\beta_0)-f(\alpha_0,\beta_1)+f(\alpha_0,\beta_0)\\
			&=[f(\alpha_1,\beta_1)-f(\alpha_1,\beta_0)]-[f(\alpha_0,\beta_1)-f(\alpha_0,\beta_0)]=R\\
			\therefore Q&=R
		\end{align*}
		Thus
		\begin{align}
			\notag\left(\eval{\pdv{f}{\alpha}}_{\point{\xi_0}{\beta_0}}-\eval{\pdv{f}{\alpha}}_{\point{\xi_1}{\beta_1}}\right)\left(\alpha_1-\alpha_0\right)&=\left(\eval{\pdv{f}{\beta}}_{\point{\alpha_0}{\zeta_0}}-\eval{\pdv{f}{\beta}}_{\point{\alpha_1}{\zeta_1}}\right)\left(\beta_1-\beta_0\right)\\
			\label{proof:SYMMETRY}\leadsto\frac{\eval{\pdv*{f}{\alpha}}_{\point{\xi_0}{\beta_0}}-\eval{\pdv*{f}{\alpha}}_{\point{\xi_1}{\beta_1}}}{\beta_1-\beta_0}&=\frac{\eval{\pdv*{f}{\beta}}_{\point{\alpha_0}{\zeta_0}}-\eval{\pdv*{f}{\beta}}_{\point{\alpha_1}{\zeta_1}}}{\alpha_1-\alpha_0}
		\end{align}
		Applying the MVT again, there exists some $\xi^\star\in(\xi_0,\xi_1),\beta^\star\in(\beta_0,\beta_1)\ni$
		\begin{align*}
			\eval{\pdv{f}{\alpha}{\beta}}_{\point{\xi^\star}{\beta^\star}}&=\frac{\eval{\pdv*{f}{\alpha}}_{\point{\xi_1}{\beta_1}}-\eval{\pdv*{f}{\alpha}}_{\point{\xi_0}{\beta_0}}}{\beta_1-\beta_0}\\
			\implies-\eval{\pdv{f}{\alpha}{\beta}}_{\point{\xi^\star}{\beta^\star}}&=\frac{\eval{\pdv*{f}{\alpha}}_{\point{\xi_0}{\beta_0}}-\eval{\pdv*{f}{\alpha}}_{\point{\xi_1}{\beta_1}}}{\beta_1-\beta_0}
		\end{align*}
		Similarly, there exists some $\alpha^\star\in(\alpha_0,\alpha_1),\zeta^\star\in(\zeta_0,\zeta_1)\ni$
		\begin{align*}
			\eval{\pdv{f}{\beta}{\alpha}}_{\point{\alpha^\star}{\zeta^\star}}&=\frac{\eval{\pdv*{f}{\beta}}_{\point{\alpha_1}{\zeta_1}}-\eval{\pdv*{f}{\beta}}_{\point{\alpha_0}{\zeta_0}}}{\alpha_1-\alpha_0}\\
			\implies-\eval{\pdv{f}{\beta}{\alpha}}_{\point{\alpha^\star}{\zeta^\star}}&=\frac{\eval{\pdv*{f}{\beta}}_{\point{\alpha_0}{\zeta_0}}-\eval{\pdv*{f}{\beta}}_{\point{\alpha_1}{\zeta_1}}}{\alpha_1-\alpha_0}
		\end{align*}
		Substituting back into \eqref{proof:SYMMETRY},
		\begin{align*}
			-\eval{\pdv{f}{\alpha}{\beta}}_{\point{\xi^\star}{\beta^\star}}&=-\eval{\pdv{f}{\beta}{\alpha}}_{\point{\alpha^\star}{\zeta^\star}}\\
			\implies\eval{\pdv{f}{\alpha}{\beta}}_{\point{\xi^\star}{\beta^\star}}&=\eval{\pdv{f}{\beta}{\alpha}}_{\point{\alpha^\star}{\zeta^\star}}
		\end{align*}
		Consequently, as $\alpha_1\rightarrow\alpha_0$ and $\beta_1\rightarrow\beta_0$, $\xi^\star\rightarrow\alpha_0,\beta^\star\rightarrow\beta_0,\alpha^\star\rightarrow\alpha_0$ 
		and $\zeta^\star\rightarrow\beta_0$. Since the derivatives are continuous,
		$$
			\eval{\pdv{f}{\beta}{\alpha}}_{\point{\alpha_0}{\beta_0}}=\eval{\pdv{f}{\alpha}{\beta}}_{\point{\alpha_0}{\beta_0}}
		$$
		Because $\point{\alpha_0}{\beta_0}$ is an arbitrary point in the domain, $\pdv{f}{\beta}{\alpha}=\pdv{f}{\alpha}{\beta}$ at 
		all points in the domain where the mixed partial derivatives are continuous.
	\end{proof}
\end{lemma}
\begin{defn}
	The \definedterm{nabla} operator $\nabla$ is a vector containing one partial derivative for each parameter of
	the function applied to \cite{RAPP2017137}. For some function $f:\mathbb{R}^n\rightarrow\mathbb{R}$, $\nabla f$ would be given by:
	$$
	\nabla f=\begin{bmatrix}
		\pdv*{f}{x_1}\\
		\pdv*{f}{x_2}\\
		\vdots\\
		\pdv*{f}{x_n}
	\end{bmatrix}
	$$
\end{defn}
\begin{defn}
	The directional derivative in the direction of some vector $\vec{v}$ of the function $f$ which is differentiable in
	the open disk $\mathbb{D}_\delta(\point{x_0}{y_0})$ centred at $\point{x_0}{y_0}$ with radius $\delta>0$ is defined as
	$$
		\eval{\nabla_{\vec{v}}f}_{\point{x_0}{y_0}}=\frac{\eval{\nabla f}_{\point{x_0}{y_0}}\cdot\vec{v}}{\lVert\vec{v}\rVert}
	$$
\end{defn}