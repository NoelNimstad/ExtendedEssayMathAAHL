\newpage
\section{Research}
\subsection{Potential flow around a circular cylinder}
A cylinder of radius $L$ is placed in two-dimensional, incompressible, inviscid flow which flows in the direction of $\ihat$.
Far away from the cylinder the velocity field $\mathbf{V}$ can be described as: 
\begin{equation}\label{equation:in-infinitum}
	\mathbf{V}=U\ihat
\end{equation}
Where $U$ is some constant. Since the cylinder is impermissible, at the boundary $\mathbf{V}\cdot\nhat=0$ where the vector $\nhat$ is the unit vector normal to the surface. 

Since in this model the viscocity $\nu=0$, the flow can be modeled using the Euler equations. If the Euler equations, apply, so does Kelvin's theorem:
\begin{theorem}[Kelvin's circulation theorem]
	The circulation around a closed material loop moving with an inviscid, barotropic fluid in the presence of conservative body forces remains constant over time.\needcitation
	
	If $\Gamma$ denotes the circulation around a material loop $C(t)$ moving with the fluid, then:
	$$\materialder{\Gamma}{t}=0$$
\end{theorem}

\textit{Id est}, if the vorticity of $\mathbf{V}$ is $0$ initialy, it must remain $0$ everywhere, thus $\curl\mathbf{V}=0$. Since the flow is irrotational, $\mathbf{V}$ can be expressed as $\mathbf{V}=\nabla\phi$, where $\phi$ is the velocity potential.

Furthermore, if $\mathbf{V}$ is incompressible, that bieng that $\divergence\mathbf{V}=0$, then $\phi$ must satisfy Laplace's equation: $\laplacian{\phi}=0$.

\subsection{Polar coordinate boundary conditions}
\subsubsection{$\mathbf{V}=U\ihat$}\label{sec:vuihat}
In polar coordinates, the base vectors $\rhat$ and $\thetahat$ are defined as:
\begin{align*}
	\rhat&\definedas\ihat\cos\theta+\jhat\sin\theta\\
	\thetahat&\definedas-\ihat\sin\theta+\jhat\cos\theta
\end{align*}
Solving for $\ihat$ and $\jhat$ gives:
\begin{align}
	\label{poihat:1}\ihat&=\frac{\rhat-\jhat\sin\theta}{\cos\theta}\\
	\label{poihat:2}\jhat&=\frac{\thetahat+\ihat\sin\theta}{\cos\theta}
\end{align}
Substituting \ref{poihat:2} into \ref{poihat:1} and isolating $\ihat$ shows that
\begin{align}
	\notag\ihat&=\frac{\rhat-\frac{\thetahat+\ihat\sin\theta}{\cos\theta}\sin\theta}{\cos\theta}\\
	\notag&=\frac{\rhat}{\cos\theta}-\frac{\thetahat\sin\theta+\ihat\sin^2\theta}{\cos^2\theta}\\
	\notag&=\frac{\rhat}{\cos\theta}-\frac{\thetahat\sin\theta}{\cos^2\theta}-\frac{\ihat\sin^2\theta}{\cos^2\theta}\\
	\notag\implies\ihat+\frac{\sin^2\theta}{\cos^2\theta}\ihat&=\frac{\rhat}{\cos\theta}-\frac{\thetahat\sin\theta}{\cos^2\theta}\\
	\notag\ihat\left(1+\frac{\sin^2\theta}{\cos^2\theta}\right)&=\frac{\rhat}{\cos\theta}-\frac{\thetahat \sin\theta}{\cos^2\theta}\\
	\notag\ihat\left(\frac{\sin^2\theta+\cos^2\theta}{\cos^2\theta}\right)&=\frac{\rhat}{\cos\theta}-\frac{\thetahat \sin\theta}{\cos^2\theta}\\
	\notag\frac{\ihat}{\cos^2\theta}&=\frac{\rhat}{\cos\theta}-\frac{\thetahat \sin\theta}{\cos^2\theta}\\
	\label{poihat:3}\ihat&=\rhat\cos\theta-\thetahat\sin\theta\qed
\end{align}
The condition stated in \ref{equation:in-infinitum} was that \textit{in infinitum}, $\mathbf{V}=U\ihat$\,. By substituting in \ref{poihat:3}, the statement becomes in terms of $\rhat$ and $\thetahat$:
$$
	\mathbf{V}=U(\rhat\cos\theta-\thetahat\sin\theta)\quad\text{as}\quad r\rightarrow\infty
$$

\subsubsection{$\mathbf{V}\cdot\nhat=0$}\label{sec:vdotnhatzero}
In polar coordinates, the base vector $\rhat$ points in the direction of positive change of $r$, that being outwards from the center. If the cylinder is assumed to be the center of the coordinate system, then $\rhat$ will always point normal to the surface of the cylinder. Therefore, at the boundary of the cylinder when $r=L$,
$$\begin{matrix}
	\mathbf{V}\cdot\rhat=0
\end{matrix}$$

\subsubsection{$\laplacian{\phi}=0$}\label{sec:deltaphizero}
% Come up with a better name later
\begin{lemma}[Jacobian Shmaycobian]
	The derivative of composite functions corresponds to the product Jacobian of Jacobian matrices:
	$$J_{f\circ g}=(J_f\circ g)J_g$$
\end{lemma}
\begin{proof}
	The Jacobian of $J_{f\circ g}$ is defined as
	$$
		J_{f\circ g}=
		\begin{bmatrix}
			\nabla^\top(f\circ g)_1 \\
			\nabla^\top(f\circ g)_2 \\
			\vdots				    \\
			\nabla^\top(f\circ g)_m
		\end{bmatrix}
	$$
	for the composite function $(f\circ g):\mathbb{R}^n\rightarrow\mathbb{R}^m$, where
	\begin{align*}		
		\row{i}J_{f\circ g}=\nabla^\top(f\circ g)_i&=\begin{bmatrix}
			\pdv{(f\circ g)_i}{x_1} & \pdv{(f\circ g)_i}{x_2} & \hdots & \pdv{(f\circ g)_i}{x_n}
		\end{bmatrix}\\
		\col{i}J_{f\circ g}&=\begin{bmatrix}
			\pdv*{(f\circ g)_1}{x_i} \\ \pdv*{(f\circ g)_2}{x_i} \\ \vdots \\ \pdv*{(f\circ g)_m}{x_i}
		\end{bmatrix}
	\end{align*}
\end{proof}

\begin{lemma}[Multivariable chain rule]
Let $X(t,u)$ and $Y(t,u)$ be functions where $X,Y:\mathbb{R}^2\rightarrow\mathbb{R}$ such that $X,Y\in C^1(\mathbb{R}^2)$. Then define $Z(x,y)$ to be a function where $Z:\mathbb{R}^2\rightarrow\mathbb{R}$ and $Z\in C^1(\mathbb{R}^2)$. Then the partial derivatives of the composite function $z(t,u)=Z(X(t,u),Y(t,u))$ are given by:
\begin{align*}
	\pdv{z}{t}&=\pdv{Z}{x}\pdv{X}{t}+\pdv{Z}{y}\pdv{Y}{t}\\
	\pdv{z}{u}&=\pdv{Z}{x}\pdv{X}{u}+\pdv{Z}{y}\pdv{Y}{u}
\end{align*}
\end{lemma}
\begin{proof}
	To be proven...
\end{proof}

\begin{lemma}[Polar-Form Laplacian]
	For some scalar field $\phi(x,y)$ defined in a Cartesian system, the Laplacian of $\phi$ in polar coordinates $\langle r,\theta\rangle$ is given by:
	$$
	\laplacian{\phi}=\pdv[2]{\phi}{r}+\frac{1}{r}\pdv{\phi}{r}+\frac{1}{r^2}\pdv[2]{\phi}{\theta}
	$$
\end{lemma}
\begin{proof}
	In Cartesian coordinates, the Laplacian operator $\laplacian$ is defined as $\nabla\cdot\nabla$, which for the scalar field $\phi$ becomes:
	\begin{align*}
		\notag\laplacian{\phi}&=\div\nabla\phi\\
							  &=\begin{pmatrix}\pdv*{x}\\\pdv*{y}\end{pmatrix}\vdot\begin{pmatrix}\pdv*{\phi}{x}\\\pdv*{\phi}{y}\end{pmatrix}\\
							  &=\pdv[2]{\phi}{x}+\pdv[2]{\phi}{y}
	\end{align*}
	Translating $x$ and $y$ to polar coordinates and calculating their derivatives with respect to $r$ and $\theta$ gives:
	\begin{align}
		\notag x=r\cos\theta&,\quad y=r\sin\theta\\
		\label{polap:1}\pdv{x}{r}=\cos\theta&,\quad\pdv{y}{r}=\sin\theta\\
		\label{polap:2}\pdv{x}{\theta}=-r\sin\theta&,\quad\pdv{y}{\theta}=r\cos\theta
	\end{align}
	Consequently, by the chain rule and substitution from \ref{polap:1}:
	\begin{align}
		\notag\pdv{\phi}{r}&=\pdv{\phi}{x}\pdv{x}{r}+\pdv{\phi}{y}\pdv{y}{r}\\
		\label{polap:3}&=\pdv{\phi}{x}\cos\theta+\pdv{\phi}{y}\sin\theta
	\end{align}
	Taking the derivative of \ref{polap:3} with respect to $r$ again gives:
	\begin{align}
		\notag\pdv[2]{\phi}{r}&=\pdv{}{r}\pdv{\phi}{x}\cos\theta+\pdv{}{r}\pdv{\phi}{y}\sin\theta\\
		\label{polap:4}&=\pdv{}{x}\pdv{\phi}{r}\cos\theta+\pdv{}{y}\pdv{\phi}{r}\sin\theta
	\end{align}
	Substituting \ref{polap:3} into \ref{polap:4} gives:
	\begin{align}
		\notag\pdv[2]{\phi}{r}&=\pdv{}{x}\left(\pdv{\phi}{x}\cos\theta+\pdv{\phi}{y}\sin\theta\right)\cos\theta+\pdv{}{y}\left(\pdv{\phi}{x}\cos\theta+\pdv{\phi}{y}\sin\theta\right)\sin\theta\\
		\notag&=\pdv[2]{\phi}{x}\cos^2\theta+\pdv{\phi}{x}{y}\sin\theta\cos\theta+\pdv{\phi}{y}{x}\cos\theta\sin\theta+\pdv[2]{\phi}{y}\sin^2\theta\\
		\label{polap:5}&=\pdv[2]{\phi}{x}\cos^2\theta+2\pdv{\phi}{x}{y}\sin\theta\cos\theta+\pdv[2]{\phi}{y}\sin^2\theta
	\end{align}
	Applying the same process for $\pdv{\phi}{\theta}$ with substitution from \ref{polap:2} yields:
	\begin{align}
		\notag\pdv{\phi}{\theta}&=\pdv{\phi}{x}\pdv{x}{\theta}+\pdv{\phi}{y}\pdv{y}{\theta}\\
		\label{polap:6}&=-\pdv{\phi}{x}r\sin\theta+\pdv{\phi}{y}r\cos\theta
	\end{align}
	Taking the derivative of \ref{polap:6} with respect to $\theta$ again gives:
	\begin{align}
		\notag\pdv[2]{\phi}{\theta}&=-\pdv{}{\theta}\pdv{\phi}{x}r\sin\theta+\pdv{}{\theta}\pdv{\phi}{y}r\cos\theta
	\end{align}
	Since both terms contain a product of two functions dependent on $\theta$ the product rule needs to be applied. This gives:
	\begin{align}
		\notag\pdv[2]{\phi}{\theta}&=-\pdv{\phi}{\theta}{x}r\sin\theta-\pdv{\phi}{x}r\cos\theta+\pdv{\phi}{\theta}{y}r\cos\theta-\pdv{\phi}{y}r\sin\theta\\
		\label{polap:7}&=-r\left(\pdv{\phi}{x}\cos\theta+\pdv{\phi}{y}\sin\theta\right)+r\pdv{\phi}{\theta}\left(-\pdv{}{x}\sin\theta+\pdv{}{y}\cos\theta\right)
	\end{align}
	Substituting \ref{polap:6} into \ref{polap:7} gives:
	\begin{align}
		\label{polap:8}\pdv[2]{\phi}{\theta}&=-r\left(\pdv{\phi}{x}\cos\theta+\pdv{\phi}{y}\sin\theta\right)+r\underbrace{\left(-\pdv{\phi}{x}r\sin\theta+\pdv{\phi}{y}r\cos\theta\right)\left(-\pdv{}{x}\sin\theta+\pdv{}{y}\cos\theta\right)}_{\Phi}
	\end{align}
	Expanding $\Phi$:
	\begin{align}
		\notag\Phi&=\left(-\pdv{\phi}{x}r\sin\theta+\pdv{\phi}{y}r\cos\theta\right)\left(-\pdv{}{x}\sin\theta+\pdv{}{y}\cos\theta\right)\\
		\notag&=\left(-\pdv{\phi}{x}r\sin\theta\right)\left(-\pdv{}{x}\sin\theta\right)+\left(-\pdv{\phi}{x}r\sin\theta\right)\left(\pdv{}{y}\cos\theta\right)\\
		\notag&+\left(\pdv{\phi}{y}r\cos\theta\right)\left(-\pdv{}{x}\sin\theta\right)+\left(\pdv{\phi}{y}r\cos\theta\right)\left(\pdv{}{y}\cos\theta\right)\\
		\notag&=\pdv[2]{\phi}{x}r\sin^2\theta-2\pdv{\phi}{x}{y}r\cos\theta\sin\theta+\pdv[2]{\phi}{y}r\cos^2\theta
	\end{align}
	Substituting $\Phi$ back into \ref{polap:8} gives:
	\begin{align}
		\notag\pdv[2]{\phi}{\theta}&=-r\left(\pdv{\phi}{x}\cos\theta+\pdv{\phi}{y}\sin\theta\right)+r\left(\pdv[2]{\phi}{x}r\sin^2\theta-2\pdv{\phi}{x}{y}r\cos\theta\sin\theta+\pdv[2]{\phi}{y}r\cos^2\theta\right)\\
		\notag&=r^2\left(\pdv[2]{\phi}{x}\sin^2\theta-2\pdv{\phi}{x}{y}\cos\theta\sin\theta+\pdv[2]{\phi}{y}\cos^2\theta\right)-r\left(\pdv{\phi}{x}\cos\theta+\pdv{\phi}{y}\sin\theta\right)\\
		\label{polap:9}&=r^2\left(\pdv[2]{\phi}{x}\sin^2\theta-2\pdv{\phi}{x}{y}\cos\theta\sin\theta+\pdv[2]{\phi}{y}\cos^2\theta\right)-r\pdv{\phi}{r}
	\end{align}
	Combining \ref{polap:5} and \ref{polap:9} yields:
	\begin{align}
		\notag\pdv[2]{\phi}{r}+\pdv[2]{\phi}{\theta}&=\pdv[2]{\phi}{x}\cos^2\theta+2\pdv{\phi}{x}{y}\sin\theta\cos\theta+\pdv[2]{\phi}{y}\sin^2\theta\\
		\notag&+r^2\left(\pdv[2]{\phi}{x}\sin^2\theta-2\pdv{\phi}{x}{y}\cos\theta\sin\theta+\pdv[2]{\phi}{y}\cos^2\theta\right)-r\pdv{\phi}{r}\\
		\notag\implies\notag\pdv[2]{\phi}{r}+\frac{1}{r^2}\pdv[2]{\phi}{\theta}&=\pdv[2]{\phi}{x}\cos^2\theta+\pdv[2]{\phi}{x}\sin^2\theta+\pdv[2]{\phi}{y}\cos^2\theta+\pdv[2]{\phi}{y}\sin^2\theta-\frac{1}{r}\pdv{\phi}{r}\\
		\notag&=\pdv[2]{\phi}{x}\left(\cos^2\theta+\sin^2\theta\right)+\pdv[2]{\phi}{y}\left(\cos^2\theta+\sin^2\theta\right)-\frac{1}{r}\pdv{\phi}{r}\\
		\notag&=\pdv[2]{\phi}{x}+\pdv[2]{\phi}{y}-\frac{1}{r}\pdv{\phi}{r}\\
		\notag\implies\pdv[2]{\phi}{x}+\pdv[2]{\phi}{y}&=\pdv[2]{\phi}{r}+\frac{1}{r}\pdv{\phi}{r}+\frac{1}{r^2}\pdv[2]{\phi}{\theta}\\
		\label{polap:final}\therefore\laplacian{\phi}&=\pdv[2]{\phi}{r}+\frac{1}{r}\pdv{\phi}{r}+\frac{1}{r^2}\pdv[2]{\phi}{\theta}
	\end{align}	
\end{proof}

\subsection{Ad confluōrem}
Summarized, the conditions translated to polar form in sections \ref{sec:vuihat}, \ref{sec:vdotnhatzero} and \ref{sec:deltaphizero} are:
$$\begin{matrix}
	&\mathbf{V}=U(\rhat\cos\theta-\thetahat\sin\theta)\quad&\text{as}\quad&r\rightarrow\infty\vecpadding\\
	&\mathbf{V}\cdot\rhat=0\quad&\text{when}\quad&r=L\vecpadding\\
	&\dfrac{\partial^2\phi}{\partial r^2}+\dfrac{1}{r}\dfrac{\partial\phi}{\partial r}+\dfrac{1}{r^2}\dfrac{\partial^2\phi}{\partial\theta^2}=0
\end{matrix}$$

testing hello hello!\cite{mat132-episode25}